\documentclass[11pt,a4paper]{article}
\usepackage[utf8]{inputenc}
\usepackage[T1]{fontenc}
\usepackage[french]{babel}
\usepackage{geometry}
\usepackage{hyperref}
\usepackage{listings}
\usepackage{xcolor}
\usepackage{booktabs}
\usepackage{longtable}
\usepackage{fancyhdr}
\usepackage{titlesec}

% Configuration de la page
\geometry{margin=2.5cm}

% Configuration des liens
\hypersetup{
    colorlinks=true,
    linkcolor=blue,
    filecolor=magenta,
    urlcolor=cyan,
}

% Configuration du code
\definecolor{codebg}{rgb}{0.95,0.95,0.95}
\definecolor{codeframe}{rgb}{0.8,0.8,0.8}
\definecolor{codestring}{rgb}{0.2,0.5,0.2}
\definecolor{codekeyword}{rgb}{0.1,0.1,0.6}

\lstset{
    backgroundcolor=\color{codebg},
    basicstyle=\ttfamily\small,
    breaklines=true,
    frame=single,
    rulecolor=\color{codeframe},
    showstringspaces=false,
    keywordstyle=\color{codekeyword},
    stringstyle=\color{codestring},
    tabsize=2,
    literate={é}{{\'e}}1 {è}{{\`e}}1 {ê}{{\^e}}1 {à}{{\`a}}1 {ù}{{\`u}}1 {û}{{\^u}}1 {ô}{{\^o}}1 {î}{{\^i}}1 {ç}{{\c{c}}}1
}

% En-tête et pied de page
\pagestyle{fancy}
\fancyhf{}
\fancyhead[L]{Budget\_HET - Chiffrage HET}
\fancyhead[R]{\thepage}
\fancyfoot[C]{Jeumont Electric}

% Titre
\title{\textbf{Budget\_HET - Chiffrage HET}\\[0.5em]\large Documentation Utilisateur}
\author{Jeumont Electric}
\date{\today}

\begin{document}

\maketitle
\tableofcontents
\newpage

% ============================================================================
\section{Introduction}

Application desktop de chiffrage d'heures d'études pour projets industriels (Jeumont Electric), développée en Python avec PyQt6. Remplace le fichier Excel BUDGET\_HET.

% ============================================================================
\section{Installation}

\subsection{Prérequis}
Python 3.10+ (3.10 ou supérieur) est requis.

\subsection{Installation des dépendances}
\begin{lstlisting}[language=bash]
pip install -r requirements.txt
# ou
pip install PyQt6
\end{lstlisting}

\subsection{Lancement}
\begin{lstlisting}[language=bash]
python main.py
\end{lstlisting}

% ============================================================================
\section{Utilisation}

\subsection{Flux de travail}

\begin{enumerate}
    \item \textbf{Onglet Général} $\rightarrow$ Remplir les informations projet (Client, CRM, DAS, type de produit, quantité, révision) puis \textbf{valider} pour charger les données par défaut.
    \item \textbf{Onglet Tâches} $\rightarrow$ Ajuster les heures d'ingénierie par tâche selon le projet.
    \item \textbf{Onglet Calculs} $\rightarrow$ Sélectionner et ajuster les heures de calcul (électromagnétique, mécanique, aéraulique).
    \item \textbf{Onglet LPDC} $\rightarrow$ Sélectionner les documents contractuels obligatoires/optionnels selon le DAS.
    \item \textbf{Onglet Options} $\rightarrow$ Activer les options techniques par catégorie (ATEX, instrumentation, essais...).
    \item \textbf{Onglet Labo} $\rightarrow$ Définir les heures d'essais laboratoire (métallurgie, isolation).
    \item \textbf{Onglet Résumé} $\rightarrow$ Consulter le total final en temps réel, avec coefficient REX et pourcentage divers.
\end{enumerate}

\subsection{Description des onglets}

\begin{longtable}{@{}lll@{}}
\toprule
\textbf{Onglet} & \textbf{Fonction} & \textbf{Données associées} \\
\midrule
\endhead
Général & Saisie des informations projet & \texttt{base\_data.json} \\
Tâches & Heures d'ingénierie de base & \texttt{general\_task\_data\_new.json} \\
Calculs & Heures de calcul par catégorie & \texttt{calculs.json} \\
LPDC & Documents contractuels & \texttt{LPDC.json} \\
Options & Catalogue d'options techniques & \texttt{options.json} \\
Labo & Heures d'essais laboratoire & \texttt{labo.json} \\
Résumé & Total des heures en temps réel & — \\
\bottomrule
\end{longtable}

\subsection{Modifier les fichiers de données}

Tous les fichiers de données sont situés dans le dossier \texttt{data/}. Ils sont au format JSON et peuvent être édités avec n'importe quel éditeur de texte.

\subsubsection{\texttt{base\_data.json} — Données de base}

Contient les listes de référence utilisées dans l'onglet Général :

\begin{lstlisting}[language=json]
{
  "people": ["Walid BOUGHANMI", "Thérèse VANDEWYNCKEL", ...],
  "product_types": {
    "SYNCH": "Synchrone",
    "ASYNCH": "Asynchrone",
    "MIL": "Marine militaire"
  },
  "products": {
    "SYNCH": { "ALT_2P": "Alternateur 2p", ... },
    "ASYNCH": { "ASYNCH_CAGE": "Asynchrone à cage", ... },
    "MIL": { "MEP": "MEP", "ANR/DAR": "ANR/DAR", "TAR": "TAR" }
  },
  "types_affaire": { "NEUF": "Machine neuve", ... },
  "DAS": { "MS": "Machines speciales", "NUC": "Nucleaire", "MIL": "Marine militaire" },
  "sectors": { ... },
  "jobs": { ... },
  "job_suffixes": { "DEF": "Definition", "PROD": "Production" }
}
\end{lstlisting}

\textbf{Pour ajouter un nouveau client} : ajouter une entrée dans \texttt{"people"}.

\textbf{Pour ajouter un nouveau type de produit} : ajouter une clé dans \texttt{"product\_types"} et les produits correspondants dans \texttt{"products"}.

\subsubsection{\texttt{general\_task\_data\_new.json} — Tâches d'ingénierie}

Structure hiérarchique des tâches avec heures de base par type de machine :

\begin{lstlisting}[language=json]
{
  "tasks": {
    "Enclenchement et Suivi": {
      "Enclenchement": {
        "reunion": {
          "base": {
            "ALT_2P": 4, "ALT_4P_L": 4, ...,
            "MEP": 700, "ANR/DAR": 500, "TAR": 600
          },
          "coeff_secteur": { "INDUS": 1, "OIL_GAS": 2, ... },
          "ortems_repartition": { "RESP_ET_PROJ_DEF": 1 }
        }
      }
    }
  }
}
\end{lstlisting}

\textbf{Propriétés d'une tâche} :
\begin{itemize}
    \item \texttt{base} : Heures de base par type de produit (ALT\_2P, ASYNCH\_CAGE, MEP, etc.)
    \item \texttt{coeff\_secteur} : Multiplicateurs par secteur (INDUS, OIL\_GAS, NUC\_QUALIFIE, etc.)
    \item \texttt{coeff\_type\_affaire} : Multiplicateurs par type d'affaire (NEUF, IDENTIQUE\_JE, REMPLACEMENT)
    \item \texttt{is\_multiplicative} : Si \texttt{true}, les heures sont multipliées par la quantité de machines
    \item \texttt{ortems\_repartition} : Répartition par métier pour l'export ORTEMS
\end{itemize}

\subsubsection{\texttt{calculs.json} — Tâches de calcul}

Liste des calculs avec catégories et règles de sélection :

\begin{lstlisting}[language=json]
{
  "categories": {
    "ELECMAG": "Electromagnetique",
    "BOB": "Bobinage",
    "CONCEPT_MECA": "Conception mecanique",
    "MECA_ANSYS": "Mecanique ANSYS",
    "AERO_THERMIQUE": "Aeraulique et Thermique"
  },
  "calculs": [
    {
      "index": 6,
      "label": "Calcul magnetique et thermique machine",
      "category": "ELECMAG",
      "hours": { "ASYNCH": 8, "SYNCH": 8 },
      "selection": {
        "ASYNCH": "mandatory",
        "SYNCH": "mandatory",
        "MIL": "optional"
      }
    }
  ]
}
\end{lstlisting}

\textbf{Règles de sélection} :
\begin{itemize}
    \item \texttt{"mandatory"} : Le calcul est automatiquement sélectionné
    \item \texttt{"optional"} : Le calcul est disponible mais non sélectionné par défaut
    \item Absence de clé : Le calcul n'est pas disponible pour ce type de produit
\end{itemize}

\subsubsection{\texttt{options.json} — Catalogue d'options}

Options groupées par catégorie avec heures associées :

\begin{lstlisting}[language=json]
{
  "categories": {
    "ATEX": "Certification ou ATEX",
    "INSTRUM": "Instrumentation et accessoires",
    "ESSAIS": "Essais"
  },
  "options": {
    "ATEX": [
      { "index": 12, "label": "ATEX Expb (Z1)", "hours": 50 },
      { "index": 13, "label": "ATEX Expzc (Z2)", "hours": 50 }
    ]
  }
}
\end{lstlisting}

\textbf{Pour ajouter une option} : ajouter un objet dans le tableau de la catégorie concernée avec un \texttt{index} unique.

\subsubsection{\texttt{LPDC.json} — Documents contractuels}

Liste des documents par DAS avec règles de sélection obligatoire/optionnelle.

\subsubsection{\texttt{labo.json} — Essais laboratoire}

\begin{lstlisting}[language=json]
{
  "categories": { "LAB_METAL": "Labo metallurgie", "LAB_ISOL": "Labo isolant" },
  "labo": [
    { "index": 1, "label": "Labo Metalo", "category": "LAB_METAL", "hours": 0 },
    { "index": 2, "label": "Labo isolant", "category": "LAB_ISOL", "hours": 0 }
  ]
}
\end{lstlisting}

% ============================================================================
\section{Notes et remarques}

\subsection{Travaux en cours}

\begin{itemize}
    \item Finaliser l'onglet Labo
    \item Créer et sauvegarder les autres projets militaires
    \item Vérifier les paramètres hydrauliques (à discuter avec l'équipe)
    \item Pour l'export ORTEMS : mettre la logique de calculs dans le Python
\end{itemize}

\subsection{Erreurs identifiées dans les fichiers Excel d'origine}

\begin{center}
\fbox{\parbox{0.9\textwidth}{
\textbf{⚠️ Important :} Ces erreurs existaient dans le fichier Excel BUDGET\_HET original et sont corrigées dans cette application.
}}
\end{center}

\subsubsection{Erreur LPDC (asynchrone)}
\begin{itemize}
    \item \textbf{Localisation} : \texttt{'tri1 LPDC'C2:I2}
    \item \textbf{Condition erronée} : \texttt{'tri calculs'C4="asynchrone"} et \texttt{'tri calculs'D4="asynchrone"} toujours \textbf{FAUX} (mauvaise référence de cellule)
    \item \textbf{Impact} : Résultats LPDC incorrects pour les machines asynchrones
\end{itemize}

\subsubsection{Erreur tâches réunion (Marine militaire)}
\begin{itemize}
    \item \textbf{Localisation} : \texttt{'chiffrage'D19} (tâches réunion)
    \item \textbf{Problème} : Référence \texttt{'data produit'!\$D8:\$M8} au lieu de \texttt{'data produit'!\$D2:\$M2}
    \item \textbf{Impact} : La ligne «~réunion~» était une copie de «~Tracé d'ensemble (3D)~» au lieu des heures de réunion
\end{itemize}

% ============================================================================
\section{Détails techniques}

\subsection{Architecture MVC}

\begin{lstlisting}
main.py                      # Point d'entree
config.xml                   # Configuration (chemins, theme UI, dimensions fenetre)
src/
  model.py                   # Classe Model (Project + calculs agreges)
  view.py                    # MainWindow (PyQt6)
  controller.py              # Controleur principal, instancie les onglets
  tabs/                      # Un controleur par onglet
  utils/
    TabTasks.py              # Table Widget et view commun
    BaseTaskTabController.py # Classe abstraite de controleur
    ApplicationData.py       # Chargement et parsing des donnees
    Task.py                  # Dataclasses : GeneralTask, LPDCDocument, Option, Calcul, Labo
data/
  base_data.json             # Listes de reference
  general_task_data_new.json # Taches d'ingenierie
  LPDC.json                  # Documents contractuels
  options.json               # Catalogue des options
  calculs.json               # Taches de calcul
  labo.json                  # Taches laboratoire
template/
  ortems_template.xlsx       # Template Excel pour l'export ORTEMS
\end{lstlisting}

\subsection{Configuration (\texttt{config.xml})}

\begin{lstlisting}[language=xml]
<?xml version="1.0" encoding="utf-8"?>
<config>
    <datapaths>
        <path key="base_data">data/base_data.json</path>
        <path key="options">data/options.json</path>
        <path key="tasks">data/general_task_data_new.json</path>
        <path key="LPDC">data/LPDC.json</path>
        <path key="calculs">data/calculs.json</path>
        <path key="labo">data/labo.json</path>
    </datapaths>
    <asset-dir>assets/</asset-dir>
    <ortems-template-path>template/ortems_template.xlsx</ortems-template-path>
    <ui>
        <theme>Fusion</theme>
        <stylesheet>src/styles.qss</stylesheet>
        <window>
            <title>Chiffrage HET</title>
            <width>1080</width>
            <height>720</height>
        </window>
    </ui>
</config>
\end{lstlisting}

\subsection{Export ORTEMS}

L'application peut exporter les heures vers un fichier Excel compatible ORTEMS via le template \texttt{template/ortems\_template.xlsx}.

La répartition par métier (job) est définie dans les fichiers JSON via la propriété \texttt{ortems\_repartition}. Exemple :

\begin{lstlisting}[language=json]
"ortems_repartition": {
  "PROJ_MACHINE_DEF": 0.7,
  "ING_MEC_SOL_DEF": 0.3
}
\end{lstlisting}

Les codes métiers disponibles sont définis dans \texttt{base\_data.json} (\texttt{jobs} et \texttt{job\_suffixes}).

\end{document}
